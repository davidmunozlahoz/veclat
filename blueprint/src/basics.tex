\chapter{The basics}

In this chapter we introduce vector lattices and prove some basic
facts such as the Riesz decomposition theorem.

\section{Vector lattices}

\begin{definition}
    \lean{VectorLattice}
    \leanok
    A \emph{vector lattice} is a real vector space $X$ together with a
    lattice order $\le $ (i.e., a partial order for which any pair of
    elements has a supremum and an infimum) satisfying:
    \begin{enumerate}
        \item if $x\le y$ and $z \in X$, then $x+z\le y+z$;
        \item if $x\le y$ and $\lambda \in \R_+$, then $\lambda x\le
            \lambda y$.
    \end{enumerate}
\end{definition}

In this chapter, $X$ will always denote a vector lattice.

\begin{definition}
    \lean{posp,negp}
    \leanok
    For every $x \in X$, we define its \emph{positive part} as
    $x_+=x\vee 0$, its \emph{negative part} as $x_-=(-x)\vee 0$, and its
    \emph{absolute value} as $|x|=x\vee (-x)$.
\end{definition}

\begin{proposition}
    \lean{nonneg_smul_sup,nonneg_smul_inf}
    \leanok
    For every $x,y \in X$ and $a \in \R_+$ the following hold:
    \[
    a(x\vee y)=(ax)\vee (ay)\text{ and }a(x\wedge y)=(ax)\wedge (ay).
    \]
\end{proposition}
\begin{proof}
    For $a=0$ the result is direct, so assume $a>0$.
    Since $x\le x\vee y$, we have $ax \le a(x\vee y)$.
    Similarly, $ay \le a(x\vee y)$. Therefore $(ax)\vee (ay)\le
    a(x\vee y)$. To prove the reverse inequality, note that
    $x=a^{-1}ax\le a^{-1}[(ax)\vee (ay)]$, and since the same is true
    of $y$, it follows that
    \[
        x\vee y\le a^{-1}[(ax)\vee (ay)].
    \]
    Multiplying both sides by $a$, $a(x\vee y)\le (ax)\vee (ay)$. The
    equality for the infimum follows from the identity $x\vee
    y=-(-x)\wedge (-y)$ that is true in every lattice ordered group.
\end{proof}

Now we can explore some properties of the positive and negative parts.

\begin{proposition}
    \label{prop:posp_negp}
    \lean{posp_sub_neg,posp_add_neg}
    \leanok
    For every $x \in X$:
    \[
    x=x_+ - x_- \text{ and }|x|=x_+ + x_-.
    \]
\end{proposition}
\begin{proof}
    For every $a$ and $b$ in a lattice ordered group, $a+b=a\vee
    b+a\wedge b$. Putting $a=x$ and $b=0$:
    \[
    x=x\vee 0 + x\wedge 0=x_+ - (-x)\vee 0=x_+ - x_-.
    \]
    For the absolute value, compute:
    \begin{align*}
        x_+ + x_- &= 2x_+ - x\\
                  &=(2x)\vee 0 - x\\
                  &=x\vee (-x)
                  &=|x|,
    \end{align*}
    where we are using that $c+a\vee b=(c+a)\vee (c+b)$ holds for
    every $a,b,c$ in a lattice ordered group.
\end{proof}

\begin{lemma}
    \lean{leq_posp_negp}
    \leanok
    For every $x,y \in X$, $x\le y$ if and only if both $x_+\le y_+$
    and $y_-\le x_-$.
\end{lemma}

\begin{lemma}
    \lean{sup_of_dis_eq_sum}
    \leanok
    Let $x,y \in X$ be such that $x\wedge y=0$. Then $x+y=x\vee y$.
\end{lemma}

The positive part (and therefore the negative part) is characterized
by the following property.

\begin{proposition}
    \lean{uniqueness_posp}
    \leanok
    Let $x,u,v \in X$ be such that $x=u-v$ and $u\wedge v=0$. Then
    $u=x_+$.
\end{proposition}

Next we provide some properties of the absolute value. All of them are
already in \verb|mathlib| but, for some reason, the first and the
second are only proved under the assumption that the space is totally
ordered.

\begin{lemma}
    \lean{abs_eq_zero_iff_zero,abs_smul',Mathlib.Algebra.Order.Group.Unbundled.Abs}
    \leanok
    For every $x,y \in X$ and $a \in \R$:
    \begin{enumerate}
        \item $|x|=0$ if and only if $x=0$;
        \item $|ax| = |a| |x|$;
        \item $|x+y|\le |x| + |y|$.
    \end{enumerate}
\end{lemma}
% \begin{proof}
%     If $x=0$, clearly $|x|=0$. Suppose $|x|=0$, by
%     \cref{prop:posp_negp} we have $0\le x_+=-x_-\le 0$, so that
%     $x_+=x_-=0$ and, again by \cref{prop:posp_negp}, $x=0$.
% \end{proof}

To prove the Riesz decomposition theorem, we will need the following
fact.

\begin{lemma}
    \lean{sub_inf_posp_sub}
    \leanok
    For every $x,y \in X$,
    \[
    x-x\wedge y=(x-y)_+.
    \]
\end{lemma}

\begin{theorem}[Riesz decomposition]
    \lean{Riesz_decomposition}
    \leanok
    Let $x,y,z \in X$ be positive elements satisfying $x\le y+z$. Then
    there exist $0\le x_1\le y$ and $0\le x_2\le z$ such that
    $x=x_1+x_2$.
\end{theorem}

From this point on, we will mostly deal with Archimedean vector
lattices.

\begin{definition}
    A vector lattice $X$ is said to be \emph{Archimedean} if for every
    $x,y \in X$ with $y>0$ there exists a natural number $n \in \N$
    such that $x\le ny$.
\end{definition}

\begin{lemma}
    \lean{arch'}
    \leanok
    Let $X$ be an Archimedean vector lattice. If $x,y \in X$ are such
    that $nx\le y$ for all $n \in \N$, then $x\le 0$.
\end{lemma}
