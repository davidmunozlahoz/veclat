\chapter{The basics}

In this chapter we introduce vector lattices, vector sublattices, and
ideals, and prove some basic facts.

\section{Vector lattices}

\begin{definition}
    \lean{VectorLattice}
    \leanok
    A \emph{vector lattice} is a real vector space $X$ together with a
    lattice order $\le $ (i.e., a partial order for which any pair of
    elements has a supremum and an infimum) satisfying:
    \begin{enumerate}
        \item if $x\le y$ and $z \in X$, then $x+z\le y+z$;
        \item if $x\le y$ and $\lambda \in \R_+$, then $\lambda x\le
            \lambda y$.
    \end{enumerate}
\end{definition}

In this chapter, $X$ will always denote a vector lattice.

\begin{definition}
    \lean{posp,negp}
    \leanok
    For every $x \in X$, we define its \emph{positive part} as
    $x_+=x\vee 0$, its \emph{negative part} as $x_-=(-x)\vee 0$, and its
    \emph{absolute value} as $|x|=x\vee (-x)$.
\end{definition}

\begin{proposition}
    \lean{nonneg_smul_sup,nonneg_smul_inf}
    \leanok
    For every $x,y \in X$ and $a \in \R_+$ the following hold:
    \[
    a(x\vee y)=(ax)\vee (ay)\text{ and }a(x\wedge y)=(ax)\wedge (ay).
    \]
\end{proposition}
\begin{proof}
    For $a=0$ the result is direct, so assume $a>0$.
    Since $x\le x\vee y$, we have $ax \le a(x\vee y)$.
    Similarly, $ay \le a(x\vee y)$. Therefore $(ax)\vee (ay)\le
    a(x\vee y)$. To prove the reverse inequality, note that
    $x=a^{-1}ax\le a^{-1}[(ax)\vee (ay)]$, and since the same is true
    of $y$, it follows that
    \[
        x\vee y\le a^{-1}[(ax)\vee (ay)].
    \]
    Multiplying both sides by $a$, $a(x\vee y)\le (ax)\vee (ay)$. The
    equality for the infimum follows from the identity $x\vee
    y=-(-x)\wedge (-y)$ that is true in every lattice ordered group.
\end{proof}

Now we can explore some properties of the positive and negative parts.

\begin{proposition}
    \label{prop:posp_negp}
    \lean{posp_sub_negp,posp_add_negp}
    \leanok
    For every $x \in X$:
    \[
    x=x_+ - x_- \text{ and }|x|=x_+ + x_-.
    \]
\end{proposition}
\begin{proof}
    For every $a$ and $b$ in a lattice ordered group, $a+b=a\vee
    b+a\wedge b$. Putting $a=x$ and $b=0$:
    \[
    x=x\vee 0 + x\wedge 0=x_+ - (-x)\vee 0=x_+ - x_-.
    \]
    For the absolute value, compute:
    \begin{align*}
        x_+ + x_- &= 2x_+ - x\\
                  &=(2x)\vee 0 - x\\
                  &=x\vee (-x)
                  &=|x|,
    \end{align*}
    where we are using that $c+a\vee b=(c+a)\vee (c+b)$ holds for
    every $a,b,c$ in a lattice ordered group.
\end{proof}

\begin{lemma}
    \lean{leq_posp_negp}
    \leanok
    For every $x,y \in X$, $x\le y$ if and only if both $x_+\le y_+$
    and $y_-\le x_-$.
\end{lemma}

\begin{lemma}
    \lean{sup_of_dis_eq_sum}
    \leanok
    Let $x,y \in X$ be such that $x\wedge y=0$. Then $x+y=x\vee y$.
\end{lemma}

The positive part (and therefore the negative part) is characterized
by the following property.

\begin{proposition}
    \lean{uniqueness_posp}
    \leanok
    Let $x,u,v \in X$ be such that $x=u-v$ and $u\wedge v=0$. Then
    $u=x_+$.
\end{proposition}

Next we provide some properties of the absolute value. All of them are
already in \verb|mathlib| but, for some reason, the first and the
second are only proved under the assumption that the space is totally
ordered.

\begin{lemma}
    \lean{abs_eq_zero_iff_zero,abs_smul'}
    \leanok
    For every $x,y \in X$ and $a \in \R$:
    \begin{enumerate}
        \item $|x|=0$ if and only if $x=0$;
        \item $|ax| = |a| |x|$;
        \item $|x+y|\le |x| + |y|$.
    \end{enumerate}
\end{lemma}
% \begin{proof}
%     If $x=0$, clearly $|x|=0$. Suppose $|x|=0$, by
%     \cref{prop:posp_negp} we have $0\le x_+=-x_-\le 0$, so that
%     $x_+=x_-=0$ and, again by \cref{prop:posp_negp}, $x=0$.
% \end{proof}

We are now in position to prove the Riesz decomposition theorem.

\begin{theorem}[Riesz decomposition]
    \lean{Riesz_decomposition}
    \leanok
    Let $x,y,z \in X$ be positive elements satisfying $x\le y+z$. Then
    there exist $0\le x_1\le y$ and $0\le x_2\le z$ such that
    $x=x_1+x_2$.
\end{theorem}

From this point on, we will mostly deal with Archimedean vector
lattices.

\begin{definition}
    A vector lattice $X$ is said to be \emph{Archimedean} if for every
    $x,y \in X$ with $y>0$ there exists a natural number $n \in \N$
    such that $x\le ny$.
\end{definition}

\begin{lemma}
    \lean{arch'}
    \leanok
    Let $X$ be an Archimedean vector lattice. If $x,y \in X$ are such
    that $nx\le y$ for all $n \in \N$, then $x\le 0$.
\end{lemma}

\section{Vector sublattices and ideals}

There are two substructures of a vector lattice that will be of
interest to us: vector sublattices and ideals.

\begin{definition}
    \lean{VectorSublattice}
    \leanok
    A \emph{vector sublattice} of $X$ is a vector subspace $Y$ that is at
    the same time a sublattice (i.e., if $x,y \in Y$, then $x\vee y
    \in Y$ and $x\wedge y \in Y$).
\end{definition}

\begin{lemma}
    \lean{VectorSublattice.abs_mem,VectorSublattice.ofSubmoduleAbs}
    \leanok
    A vector subspace of $X$ is a vector sublattice if and only if it
    is closed under taking absolute values.
\end{lemma}

\begin{definition}
    \lean{VectorOrderIdeal}
    \leanok
    A \emph{vector order ideal} $I$ in $X$ is a vector sublattice such that, if
    $|x|\le |y|$ and $y \in I$, then $x \in I$.
\end{definition}

In the literature, vector order ideals are called order ideals or just
ideals. When no risk of confusion arises, we shall do the same.
However, in Lean, we have to be more careful. Ideal refers to an
algebraic ideal, and order ideal refers to an ideal in a partially
ordered set. Hence we have to use the full name.

\begin{lemma}
    \lean{VectorOrderIdeal.mem_iff_abs_mem}
    \leanok
    Let $I$ be an ideal in $X$. Then $x \in I$ if and only if $|x| \in I$.
\end{lemma}

% \section{Principal ideals}
%
% Of special interest is the ideal generated by a single element.
%
% \begin{definition}
%     \lean{PrincipalIdeal}
%     \leanok
%     Let $a \in X_+$. The \emph{principal ideal generated by $a$} is
%     \[
%     I_a = \{\, x \in X : |x|\le s a\text{ for some }s \in \R_+\, \}.
%     \]
% \end{definition}
%
% \begin{lemma}
%     \lean{PrincipalIdeal.instVectorOrderIdeal}
%     \leanok
%     The principal order ideal generated by an element is a vector
%     order ideal.
% \end{lemma}
%
% From now on, let $X$ be Archimedean and let $a \in X_+$ be fixed.
%
% \begin{definition}
%     \lean{PrincipalIdeal.S,PrincipalIdeal.norm}
%     \leanok
%     For every $x \in I_a$, define its associated norm by
%     \[
%         \|x\|_a = \inf\{\, s \in \R_+ : |x| \le s a \, \}.
%     \]
% \end{definition}
%
% It is an important fact that this infimum is attained.
%
% \begin{lemma}
%     \lean{PrincipalIdeal.norm_attained}
%     \leanok
%     For every $x \in I_a$, $|x|\le \|x\|_a a$.
% \end{lemma}
%
% We can now check that $\|{\cdot }\|_a$ is indeed a norm.
%
% \begin{lemma}
%     \lean{PrincipalIdeal.norm_nonneg,PrincipalIdeal.norm_zero_iff_zero,PrincipalIdeal.norm_smul,PrincipalIdeal.norm_add}
%     For every $x,y \in I_a$ and $\lambda  \in \R$, the following hold:
%     \begin{enumerate}
%         \item $\|x\|_a=0$ if and only if $x=0$;
%         \item $\|\lambda x\|_a= |\lambda | \|x\|_a$;
%         \item $\|x+y\|_a\le \|x\|_a+\|y\|_a$.
%     \end{enumerate}
% \end{lemma}
%
% In fact, it is an AM-norm.
%
% \begin{lemma}
%     \lean{PrincipalIdeal.norm_le,PrincipalIdeal.AMnorm}
%     \leanok
%     For every $x,y \in I_a$:
%     \begin{enumerate}
%         \item if $|x|\le |y|$, then $\|x\|_a\le \|y\|_a$;
%         \item if $x,y\ge 0$, then $\|x \vee y\|_a=\|x\|_a\vee \|y\|_a$.
%     \end{enumerate}
% \end{lemma}
